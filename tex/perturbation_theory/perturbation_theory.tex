\documentclass[12pt]{article}

\usepackage[T1]{fontenc}
\usepackage[utf8]{inputenc}
\usepackage[russian]{babel}

% page margin
\usepackage[top=2cm, bottom=2cm, left=2cm, right=2cm]{geometry}

% AMS packages
\usepackage{amsmath}
\usepackage{amssymb}
\usepackage{amsfonts}
\usepackage{amsthm}

\usepackage{bbm}

\usepackage{array}
\usepackage{graphicx}

\usepackage{fancyhdr}
\pagestyle{fancy}
% modifying page layout using fancyhdr
\fancyhf{}
\renewcommand{\sectionmark}[1]{\markright{\thesection\ #1}}
\renewcommand{\subsectionmark}[1]{\markright{\thesubsection\ #1}}

\rhead{\fancyplain{}{\rightmark }}
\cfoot{\fancyplain{}{\thepage }}

\usepackage{titlesec}
\titleformat{\section}{\bfseries}{\thesection.}{1em}{}
\titleformat{\subsection}{\normalfont\itshape\bfseries}{\thesubsection.}{0.5em}{}

\newcommand{\HO}{H^{(0)}}

\newcommand{\lc}{\left\{}
\newcommand{\rc}{\right\}}

\begin{document}

\section{Теория возмущений стационарных вырожденных состояний}

Рассмотрим возмущенный гамильтониан
\begin{gather}
    H(\lambda) = \HO + \lambda \delta H,
\end{gather}
где спектр собственных значений и соответствующих собственных функций невозмущенного оператора $\HO$ известен. Рассмотрим подпространство вырожденных собственных значений размерности $N > 1$, то есть, пространство, образующими которого являются $N$ линейно независимых собственных функций с равными собственными значениями. В базисе собственных функций матрица невозмущенного гамильтониана $\HO$ имеет диагональный вид
\begin{gather}
    \HO = diag \lc E_1^{(0)} \rc
\end{gather}

\end{document}
